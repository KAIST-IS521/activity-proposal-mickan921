\documentclass[a4paper, 11pt]{article}

\usepackage{kotex} % Comment this out if you are not using Hangul
\usepackage{fullpage}
\usepackage{hyperref}
\usepackage{amsthm}
\usepackage[numbers,sort&compress]{natbib}

\theoremstyle{definition}
\newtheorem{exercise}{Exercise}

\begin{document}
%%% Header starts
\noindent{\large\textbf{IS-521 Activity Proposal}\hfill
                \textbf{오정석}} \\
         {\phantom{} \hfill \textbf{mickan921}} \\
         {\phantom{} \hfill Due Date: April 15, 2017} \\
%%% Header ends

\section{Activity Overview}

뉴스에서도 자주 접할수 있는 웹해킹은 대중에게 가장 잘 알려진  사이버 보안위협중 하나 입니다. 대부분의 웹사이트는 단순 html이 아닌 java script나 angularJS 혹은 html5와 같은 언어들로 프로그래밍 되있고 이들에 따른 보안 위협또한 현존한다. 본 과제의 목적은 널리 알려전 웹해킹 기술들을 탐구하고 공격 원리를 학습하는데 있다.

\section{Exercises}

Describe a series of exercises that students will carry out. (학생들이 하게
될 연습문제를 순차적으로 서술.)

\begin{exercise}

  SQL injection 같은 기술등을 이용해서 허용되지 않은 로그인을 한다.

\end{exercise}

\begin{exercise}

  XSS (Cross Site Scripting) 을 이용해서 웹페이지에 악성 스크립트 삽입

\end{exercise}

\begin{exercise}

  CSRF (Cross Site Request Forgery)를 이용해서 사이트에 악성 요청 전송

\end{exercise}

\section{Expected Solutions}

Vulnet과 같이 공격 대상 웹페이지가 존재하고 이를 학생들이 공격하면서

웹 해킹에 대한 이해를 높힐 수 있을것 같습니다.

\bibliography{references}
\bibliographystyle{plainnat}

\end{document}
